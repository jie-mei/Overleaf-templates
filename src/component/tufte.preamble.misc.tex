\usepackage{subcaption}
\captionsetup{compatibility=false}

%-------------------------------------------------------------------------------
% Preliminary
%-------------------------------------------------------------------------------

\usepackage{xcolor}
\usepackage{amssymb} % used by glossary

\usepackage{soul} % enable `ul` command that allow line break for underline text

%-------------------------------------------------------------------------------
% Linkage
%-------------------------------------------------------------------------------

\usepackage{url}
\makeatletter
\g@addto@macro{\UrlBreaks}{\UrlOrds}
\makeatother

% Note: This package have to be placed before 'glossary' and after 'imakeidx'.
\usepackage{hyperref}
\hypersetup{
  bookmarksnumbered=true,
  bookmarksopen=true,
  bookmarksopenlevel=1,
  colorlinks=true,
  %footnotecolor=blue,
  pdfstartview=Fit,
  pdfpagemode=UseOutlines
}

%-------------------------------------------------------------------------------
% Glossary
%-------------------------------------------------------------------------------

\IfFileExists{glossary.tex}{
  \usepackage[nopostdot,toc,style=super]{glossaries}
  \makeglossaries
  \input{glossary}
}{}

%-------------------------------------------------------------------------------
% Glossary
%-------------------------------------------------------------------------------


% Used for printing a trailing space better than using a tilde (~) using the
% \xspace command
\usepackage{xspace}
% Print text in maroon
\newcommand{\hlred}[1]{\textcolor{Maroon}{#1}}
% Command to print a very short space
\newcommand{\hairsp}{\hspace{1pt}}
% Command to print i.e.
\newcommand{\ie}{\textit{i.\hairsp{}e.}\xspace}
% Command to print e.g.
\newcommand{\eg}{\textit{e.\hairsp{}g.}\xspace}

% New command for note.
\newcommand{\note}[1]{\footnote{\hlred{NOTE}: #1}}

% Set up the images/graphics package
\usepackage{graphicx}
\setkeys{Gin}{width=\linewidth,totalheight=\textheight,keepaspectratio}
\graphicspath{{graphics/}}

% The units package provides nice, non-stacked fractions and better spacing
% for units.
\usepackage{units}
\usepackage[english]{babel}
\usepackage[latin1]{inputenc}

% The fancyvrb package lets us customize the formatting of verbatim
% environments. We use a slightly smaller font.
\usepackage{fancyvrb}
\fvset{fontsize=\normalsize}

% Definition environment.
\usepackage{thmtools}
%\theoremstyle{definition}
\declaretheorem[style=remark]{remark}
\newtheorem{definition}{Definition}
\declaretheorem[style=remark]{example}

%----------------------------------------------------------------
% Footnote Format
%----------------------------------------------------------------

% Define footnote color option.
\ifx\mycounter\undefined
  \newcounter{Hfootnote}
\fi
\makeatletter
\def\@footnotecolor{red}
\define@key{Hyp}{footnotecolor}{%
 \HyColor@HyperrefColor{#1}\@footnotecolor%
}
\def\@footnotemark{%
  \leavevmode
  \ifhmode\edef\@x@sf{\the\spacefactor}\nobreak\fi
  \stepcounter{Hfootnote}%
  \global\let\Hy@saved@currentHref\@currentHref
  \hyper@makecurrent{Hfootnote}%
  \global\let\Hy@footnote@currentHref\@currentHref
  \global\let\@currentHref\Hy@saved@currentHref
  \hyper@linkstart{footnote}{\Hy@footnote@currentHref}%
  \@makefnmark
  \hyper@linkend
  \ifhmode\spacefactor\@x@sf\fi
  \relax
}%
\makeatother

%----------------------------------------------------------------
% TOC Format
%----------------------------------------------------------------

\setcounter{tocdepth}{3}

\pdfbookmark{\contentsname}{toc}

%----------------------------------------------------------------
% Section Format
%----------------------------------------------------------------

% add numbers to chapters, sections, subsections
\setcounter{secnumdepth}{2}

% chapter format
\titleformat{\chapter}%
  {\huge\rmfamily\itshape}% format applied to label+text
  {}% label
  {0em}% horizontal separation between label and title body
  {}% before the title body
  []% after the title body

% section format
\titleformat{\section}%
  {\normalfont\Large\itshape}% format applied to label+text
  {\llap{\colorbox{black}{\parbox{1.5cm}{\hfill\color{white}\thesection}}\hspace{0.5em}}}% label
  {0em}% horizontal separation between label and title body
  {}% before the title body
  []% after the title body

% subsection format
\titleformat{\subsection}%
  {\normalfont\large\itshape\color{black}}% format applied to label+text
  {\llap{\colorbox{gray}{\parbox{1.0cm}{\hfill\color{white}\thesubsection}}\hspace{0.5em}}}% label
  {0em}% horizontal separation between label and title body
  {}% before the title body
  []% after the title body

%----------------------------------------------------------------
% Math Annotations
%----------------------------------------------------------------

\newcommand{\mmatrix}[1]{\ensuremath \mathbf{#1}}
\newcommand{\mvector}[1]{\ensuremath \mathbf{#1}}
\newcommand{\mset}[1]{\ensuremath \mathcal{#1}}

\usepackage{xstring} % enable conditioning on string values

\usepackage{xcolor}

\definecolor{commentsColor}{rgb}{0.497495, 0.497587, 0.497464}
\definecolor{keywordsColor}{rgb}{0.000000, 0.000000, 0.635294}
\definecolor{stringColor}{rgb}{0.558215, 0.000000, 0.135316}

\usepackage{listings}
\lstset{ %
  backgroundcolor=\color{white},   % choose the background color; you must add \usepackage{color} or \usepackage{xcolor}
  basicstyle=\footnotesize\ttfamily,        % the size of the fonts that are used for the code
  breakatwhitespace=false,         % sets if automatic breaks should only happen at whitespace
  breaklines=true,                 % sets automatic line breaking
  captionpos=b,                    % sets the caption-position to bottom
  commentstyle=\color{commentsColor}\textit,    % comment style
  deletekeywords={...},            % if you want to delete keywords from the given language
  escapeinside={\%*}{*)},          % if you want to add LaTeX within your code
  extendedchars=true,              % lets you use non-ASCII characters; for 8-bits encodings only, does not work with UTF-8
  frame=tb,	                   	   % adds a frame around the code
  keepspaces=true,                 % keeps spaces in text, useful for keeping indentation of code (possibly needs columns=flexible)
  keywordstyle=\color{keywordsColor}\bfseries,       % keyword style
  language=Python,                 % the language of the code (can be overrided per snippet)
  otherkeywords={*,...},           % if you want to add more keywords to the set
  numbers=left,                    % where to put the line-numbers; possible values are (none, left, right)
  numbersep=5pt,                   % how far the line-numbers are from the code
  numberstyle=\tiny\color{commentsColor}, % the style that is used for the line-numbers
  rulecolor=\color{black},         % if not set, the frame-color may be changed on line-breaks within not-black text (e.g. comments (green here))
  showspaces=false,                % show spaces everywhere adding particular underscores; it overrides 'showstringspaces'
  showstringspaces=false,          % underline spaces within strings only
  showtabs=false,                  % show tabs within strings adding particular underscores
  stepnumber=1,                    % the step between two line-numbers. If it's 1, each line will be numbered
  stringstyle=\color{stringColor}, % string literal style
  tabsize=2,	                   % sets default tabsize to 2 spaces
  title=\lstname,                  % show the filename of files included with \lstinputlisting; also try caption instead of title
  columns=fixed                    % Using fixed column width (for e.g. nice alignment)
}

%----------------------------------------------------------------
% Config from Xiang
%----------------------------------------------------------------


\newcommand{\red}[1]{{\color{red}{#1}}}
\newcommand{\blue}[1]{{\color{blue}{#1}}}

\usepackage{fancyvrb}
\usepackage{mathrsfs}
\usepackage{dutchcal}
\usepackage{yfonts}
\usepackage{eufrak}

\usepackage{graphicx} % Needed to insert images into the document
\graphicspath{{img/}}
\newtheorem{prop}{Proposition}
\newtheorem{theorem}{Theorem}
